\documentclass[12pt]{article}
\usepackage{fontspec}
\usepackage{hyperref}
\usepackage{parskip}
\usepackage{setspace}
\usepackage{microtype}
\usepackage[a4paper,margin=2.25cm]{geometry}

\setmainfont{Libertinus Serif}
\setmonofont{Source Code Pro}
\setstretch{1.3}
\urlstyle{rm}

% https://tex.stackexchange.com/a/10689
\usepackage{enumitem}
\setlist{noitemsep}

\begin{document}
\pagestyle{empty}

\begin{center}

  {\Huge Luca Piras}

  \vspace{0.616cm}

  lucapiras13@protonmail.com

  \url{https://luca-piras.com} --- \url{https://github.com/lucapiras5}

  +39 339 187 1837 (cell.) --- +39 051 1998 0584 (fisso)

  \vspace{0.616cm}

\end{center}

{\Large Introduzione ed obiettivi professionali}

Sono un neolaureato in giurisprudenza, interessato a stage e tirocini extracurricolari per la formazione e l'inserimento. Sono disponibile a lavorare immediatamente, e sono disponibile a trasferte.

Ho una grande passione per l'informatica, e ho acquisito competenze da sistemista e programmatore come autodidatta.

La mia ambizione principale è di diventare un \textit{digital forensics consultant} (consulente d'informatica forense), ma nutro un interesse per l'informatica giuridica in generale.

{\Large Formazione}

\begin{itemize}
\item Maggio 2024 --- Laurea Magistrale in Giurisprudenza, conseguita presso l'università di Bologna, con voto 110 e lode e tesi sull'informatica forense: \textit{Uso del software libero e open source per l'analisi scientifica della prova digitale nell'informatica forense};\footnote{Disponibile online, con licenza CC-BY-SA 4.0: \url{https://luca-piras.com/static/pdf/Tesi_Informatica_Forense_2024.pdf} e \url{https://github.com/lucapiras5/tesi-informatica-forense}.}
\item Ottobre 2017 --- Corso di Informatica Forense presso BIT4LAW (aspetti giuridici e tecnici, simulazioni di attività tecniche).
\end{itemize}

{\Large Lingue straniere}

\begin{itemize}
\item Ottima conoscenza della lingua inglese, parlata e scritta (livello C1).
\end{itemize}

{\Large Aree del diritto in cui sono interessato a lavorare}

\begin{itemize}
\item Diritto penale e procedura penale --- in particolare, i \textit{cybercrimes} ed il trattamento della prova digitale;
\item Diritti intellettuali --- in particolare, il diritto d'autore, e le licenze d'uso del software libero;
\item Informatica giuridica --- in generale, qualsiasi intersezione tra informatica e diritto: processo civile telematico, diritto delle nuove tecnologie, trasformazione digitale, \textit{cybersecurity}, \textit{compliance} con norme nazionali ed europee, ecc.
\end{itemize}

{\Large Competenze relative all'informatica}

\begin{itemize}
\item Conoscenza delle caratteristiche tecniche della \textit{digital evidence}\footnote{V. la tesina scritta in occasione del corso di informatica forense all'università: \url{https://luca-piras.com/static/informatica-forense/Alcune_osservazioni_sulla_digital_evidence_2020.pdf}.} e delle relative norme all'interno del codice di procedura penale (v. primi due capitoli della tesi);
\item Conoscenza delle \textit{best practices} per lo sviluppo del software rilevanti per valutarne l'affidabilità (v. terzo capitolo della tesi);
\item Esperienza con l'installazione e amministrazione di distribuzioni GNU/Linux, anche come macchine virtuali, su server, e da remoto;
\item Conoscenze di base relative alle distribuzioni GNU/Linux e applicazioni \textit{open source} utilizzate per l'informatica forense (v. quarto capitolo della tesi);
\item Familiarità con (e preferenza per) l'uso di programmi con interfacce a riga di comando, e capacità di creare \textit{scripts} in Bash e Python per automatizzare lo svolgimento di operazioni e la creazione di report;
\item Conoscenza dei programmi per svolgere operazioni sui principali \textit{filesystem} (Ext4, FAT, NTFS), conoscenza di base di ZFS;
\item Conoscenze di base relative al \textit{web development}, tra cui:
  \begin{itemize}
  \item Familiarità con i protocolli di rete, in particolare HTTP(S);
  \item Esperienza di base con lo sviluppo di siti web \textit{full-stack} e l'impiego di database relazionali (SQLite, PostgreSQL);
  \item Conoscenza delle principali vulnerabilità relative alle applicazioni sul web;
  \item Creazione di un proprio sito web;\footnote{Il codice per il sito \url{https://luca-piras.com} è disponibile su \url{https://github.com/lucapiras5/luca-piras.com}.}
  \end{itemize}
\end{itemize}

{\Large \textit{Soft skills}}

\begin{itemize}
\item Capacità di lavorare per lunghi periodi e desiderio di formazione continua, dovute alla passione per l'informatica;
\item Precisione e forte attenzione ai dettagli;
\item Capacità di lavorare in autonomia per quanto riguarda la ricerca di informazioni e la risoluzione di problemi;
\item Interesse per l'ottimizzazione dei processi comunicativi ed organizzativi all'interno di un team, e l'automatizzazione del lavoro;
\item Interesse per la ricerca, organizzazione, documentazione e condivisione di informazioni e \textit{know-how};
\item Capacità di spiegare concetti tecnici ad un pubblico non specialistico in maniera concisa ed accessibile, senza compromettere la precisione tecnica.
\end{itemize}

\begin{center}
\textit{Autorizzo il trattamento dei dati personali presenti nel CV,\\ai sensi del D.Lgs. 2018/101 e del GDPR (Regolamento UE 2016/679).}
\end{center}

\end{document}
