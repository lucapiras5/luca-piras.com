\documentclass[12pt]{article}
\usepackage{fontspec}
\usepackage{hyperref}
\usepackage{parskip}
\usepackage{setspace}
\usepackage{microtype}
\usepackage[a4paper]{geometry}

\setmainfont{Libertinus Serif}
\setmonofont{Source Code Pro}
\setstretch{1.3}
\urlstyle{rm}

% https://tex.stackexchange.com/a/10689
\usepackage{enumitem}
\setlist{noitemsep}

\begin{document}
\pagestyle{empty}

\begin{center}

	{\Huge Luca Piras}

	\vspace{0.616cm}

	lucapiras13@protonmail.com

	https://lucapiras.me --- https://github.com/lucapiras5

	+39 339 187 1837 (cell.) --- +39 051 1998 0584 (fisso)

	\vspace{0.616cm}

\end{center}

{\Large Formazione}

\begin{itemize}
	\item 2024 --- Laurea Magistrale in Giurisprudenza conseguita presso l'università di Bologna con voto 110 e lode e tesi sull'informatica forense.
\end{itemize}

{\Large Informatica forense}

Le seguenti opere sono disponibili su GitHub\footnote{\url{https://github.com/lucapiras5/cv/tree/main/informatica-forense}} e sono rilasciate con licenza CC BY-SA 4.0.\footnote{\url{https://creativecommons.org/licenses/by-sa/4.0}}

\begin{itemize}
	\item \textit{Uso del software libero e open source per l'analisi scientifica della prova digitale nell'informatica forense}, 2024, tesi di laurea;\footnote{I file sorgenti sono disponibili su \url{https://github.com/lucapiras5/tesi-informatica-forense}}
	\item \textit{Alcune osservazioni sulla digital evidence -- caratteristiche, ricerca, valutazione}, 2020, report scritto durante il corso di informatica forense.
\end{itemize}

{\Large Competenze}

\begin{itemize}
	\item Ottima conoscenza della lingua inglese, parlata e scritta (livello C1);
	\item Conoscenza delle distribuzioni sistemi GNU/Linux e delle applicazioni da linea di comando;
	\item Esperienza con la gestione di \textit{filesystems}, tra cui anche ZFS.
	\item Capacità di creare \textit{scripts} in Bash e Python per automatizzare operazioni e generare report;
	\item Conoscenze di base relative al \textit{web development}: protocollo HTTP(S), uso di RDBMS (SQLite, PostgreSQL), uso di \textit{back-end frameworks} (Flask, Express), conoscenza delle principali vulnerabilità relative alle applicazioni sul web, uso di JavaScript;
	\item Capacità di usare Git, LaTeX
	\item Capacità di amministrare un server GNU/Linux da remoto con SSH
	\item Conoscenza delle \textit{best practices} relative allo sviluppo del software
\end{itemize}

{\Large \textit{Soft skills}}

\begin{itemize}
	\item Passione per l'informatica;
	\item Attenzione ai dettagli;
	\item Capacità di spiegare concetti tecnici in maniera chiara ed accessibile anche ad un pubblico non specialistico;
	\item Abilità di ricerca di informazioni e \textit{problem-solving} in autonomia;
\end{itemize}

{\Large \textit{Interessi futuri}}

\begin{itemize}
	\item Uso di container / NixOS per ambienti di analisi deterministici
	\item Live-debugging e ptrace
	\item c.a.d. e strumenti open-source
	\item normative sulla cybersicurezza
\end{itemize}

\end{document}
